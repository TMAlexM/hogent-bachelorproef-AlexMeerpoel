%==============================================================================
% Sjabloon onderzoeksvoorstel bachproef
%==============================================================================
% Gebaseerd op document class `hogent-article'
% zie <https://github.com/HoGentTIN/latex-hogent-article>

% Voor een voorstel in het Engels: voeg de documentclass-optie [english] toe.
% Let op: kan enkel na toestemming van de bachelorproefcoördinator!
\documentclass{hogent-article}

% Invoegen bibliografiebestand
\addbibresource{voorstel.bib}

% Informatie over de opleiding, het vak en soort opdracht
\studyprogramme{Professionele bachelor toegepaste informatica}
\course{Bachelorproef}
\assignmenttype{Onderzoeksvoorstel}
% Voor een voorstel in het Engels, haal de volgende 3 regels uit commentaar
% \studyprogramme{Bachelor of applied information technology}
% \course{Bachelor thesis}
% \assignmenttype{Research proposal}

\academicyear{2023-2024} % TODO: pas het academiejaar aan

% TODO: Werktitel
\title{De werking en overeenstemming van Customer Relationship Management systemen met huidige cybersecurity maatregelen, en de samenwerking met Data Loss Prevention implementatie.}

% TODO: Studentnaam en emailadres invullen
\author{Alex Meerpoel}
\email{alex.meerpoel@student.hogent.be}

% TODO: Medestudent
% Gaat het om een bachelorproef in samenwerking met een student in een andere
% opleiding? Geef dan de naam en emailadres hier
% \author{Yasmine Alaoui (naam opleiding)}
% \email{yasmine.alaoui@student.hogent.be}

% TODO: Geef de co-promotor op
\supervisor[Co-promotor]{M. Robyn (Excentis NV, \href{mailto:moreno.robyn@excentis.com}{moreno.robyn@excentis.com})}

% Binnen welke specialisatierichting uit 3TI situeert dit onderzoek zich?
% Kies uit deze lijst:
%
% - Mobile \& Enterprise development
% - AI \& Data Engineering
% - Functional \& Business Analysis
% - System \& Network Administrator
% - Mainframe Expert
% - Als het onderzoek niet past binnen een van deze domeinen specifieer je deze
%   zelf
%
\specialisation{System \& Network Administrator}
\keywords{Cybersecurity, Customer Relationship Management, Enterprise Resource Planning, Data Loss Prevention}

\begin{document}

\begin{abstract}
  Bedrijven gebruiken Customer Relationship Management (CRM) systemen om bedrijfsprocessen te vergemakkelijken. Die systemen verwerken gevoelige gegevens en moeten dus goed beveiligd zijn.
  Binnen Excentis NV wordt hiervoor gebruik gemaakt van Hubspot in samenwerking met een Sharepoint omgeving. Om dit bedrijfsproces te vergemakkelijken, worden enkele CRM alternatieven overwogen, waaronder Odoo, met een focus op cybersecurity.
  
  Welke CRM alternatieven bieden dezelfde functies als Sharepoint? Hoe voldoen CRM systemen aan huidige cybersecurity standaarden? Hoe integreert Data Loss Prevention zich in deze systemen?
  
  Om een antwoord op deze vragen te formuleren, wordt er een literatuurstudie gedaan over de huidige cybersecurity wetgeving binnen Europa en België. Daarna wordt er aan de hand van een requirements-analyse gekeken welke CRM systemen de beste beveiligingsmogelijkheden bieden.
  
  Vervolgens worden de best beveiligde systemen opgezet aan de hand van een proof-of-concept. Tenslotte wordt in deze systemen een Data Loss Prevention geïmplementeerd.
  
  Er wordt verwacht dat de meeste CRM systemen een basis van veiligheidsmaatregelen hebben. De tools die aan dit voldoen, zullen ook overeenstemmen met de huidige Europese cybersecurity richtlijnen.
  
  Het doel van het onderzoek is om een rapport op te stellen over de implementatie van een beveiligd Customer Relationship Management systeem met een geïntegreerde Data Loss Prevention. Hierbij kunnen eventuele ontbrekende best practices worden aangekaart.
  
\end{abstract}

\tableofcontents

% De hoofdtekst van het voorstel zit in een apart bestand, zodat het makkelijk
% kan opgenomen worden in de bijlagen van de bachelorproef zelf.
%---------- Inleiding ---------------------------------------------------------

\section{Introductie}%
\label{sec:introductie}

Customer Relationship Management (CRM) is voor elk bedrijf een belangrijk bedrijfsproces. Tijdens dit proces wordt er gevoelige data verwerkt en bijgehouden. Hiervoor bestaan er ondertussen al verschillende software pakketten om dit te vergemakkelijken. Het verwerken van deze data moet dan ook goed beveiligd zijn tegen cyberaanvallen.

Momenteel doet het bedrijf Excentis NV aan Customer Relationship Management met behulp van Microsoft Sharepoint, Hubspot en Odoo. Om dit bedrijfsproces te vergemakkelijken, wil het bedrijf overstappen van drie systemen naar één systeem. Hierbij is het cybersecurity aspect sterk van belang om de gevoelige customer data te beveiligen.

\newline
Volgende zaken worden onderzocht:
\begin{itemize}
    \item Hoe voldoen CRM pakketten aan de huidige cybersecurity best practices en standaarden?
    \item Hoe behandelen deze CRM systemen Data Loss Prevention (DLP)?
    \item Hoe kunnen CRM systemen integreren met systemen zoals JIRA ITSM, Opsgenie, MS Sentinel voor detectie, reactie en herstel?
\end{itemize}
  
Uit deze vragen wordt een rapport opgesteld met een concrete uitwerking over hoe Data Loss Prevention kan samenwerken met CRM systemen.

Hierbij wordt een proof-of-concept opgezet van 3 of 4 CRM mogelijkheden om aan te tonen hoe deze veilig kunnen opgezet worden en Data Loss Prevention toepassen.


%---------- Stand van zaken ---------------------------------------------------

\section{State-of-the-art}%
\label{sec:state-of-the-art}
\subsection{Cybersecurity Standaarden}

De voornamelijkste cybersecurity standaarden waaraan een bedrijf, en de processen, moet voldoen zijn de \emph{General Data Protection Regulation (GDPR)} en \emph{Network and Information Security 2 (NIS2) Directive}.
\subsubsection{GDPR}

In 2016 heeft de Europese Commissie de GDPR richtlijnen opgesteld. De richtlijnen geven de privacy rechten aan in verband met persoonsgegevens van elke Europese burger. Sinds mei 2020 wordt er om de vier jaar een evaluatie gedaan van deze regulatie. \autocite{Office2016}

De verwerking van persoonsgegevens moet voldoen aan de beginselen van de GDPR. Dit wil zeggen dat de verwerkte gegevens alleen het strikt noodzakelijke mogen bevatten, actueel moeten zijn en voor een beperkte tijd opgeslagen mogen worden. Bijkomend moeten de gegevens beschermd zijn tegen verlies of beschadiging en is de persoon of instantie die de gegevens verwerkt aansprakelijk.

\subsubsection{NIS2}
De Europese Unie heeft het NIS2 opgesteld met als doel een gelijk en hoog niveau van cybersecurity te bereiken in alle lidstaten. Deze richtlijn zou in elke lidstaat moeten geadopteerd zijn tegen 17 oktober 2024. Daarna evalueert de Europese Commissie elke drie jaar de richtlijn opnieuw om eventuele aanpassingen in te voeren. \autocite{Office2022}

NIS2 zet een aantal basismaatregelen op voor bedrijven op basis van hun grootte en werksector. Zo verwacht de Europese Unie dat elk middelgroot bedrijf een plan heeft om cyberincidenten te behandelen, alsook back-upbeheer. Waar het past moet multifactor authenticatie (MFA) gebruikt worden.

\subsubsection{Cyberfundamentals Framework}
Naar aanleiding van voorgaande richtlijnen heeft het \textcite{FUNDAMENTALS2023} (CCB) een framework opgesteld om bedrijven te helpen voldoen aan bepaalde niveaus van beveiliging. Het framework is gebaseerd op vier veelgebruikte cybersecurity frameworks; namelijk NIST CSF, ISO 27001/ISO 27002, CIS Controls en IEC 62443.

In dit framework is er een onderscheid tussen 4 niveaus van beveiliging; Small, Basis, Belangrijk en Essentieel. De richtlijnen binnen het Small niveau dienen om een eerste inschatting te maken voor micro-organisaties. Terwijl het Essentieel niveau bedoeld is om geavanceerde cyberaanvallen tegen te gaan. \autocite{FUNDAMENTALS2023a}

\subsection{Security in CRM systemen}
Volgens \textcite{CRMSecurity} zijn er een aantal cyberaanvallen die het meest een bedreiging vormen voor Customer Relationship Management systemen. \newline
Een Account Takeover (ATO) houdt in dat een onbevoegde persoon de inloggegevens van een bevoegd persoon gebruikt om in het systeem data te stelen. \newline

Een Denial of Service Attack (DoS), of Distributed Denial of Service Attack (DDoS), kan ervoor zorgen dat het CRM systeem gedurende een bepaalde tijd niet meer werkt. Dit kan gevolgen hebben voor een bedrijf, aangezien ze hierdoor tijdelijk contact verliezen met hun klanten en dus alternatieven moeten gebruiken om hun klanten in te lichten.

\subsection{Data Loss Prevention}
Data Loss Prevention systemen monitoren data in een netwerk en houden bij wie toegang heeft tot welk type data. Op die manier is access control toegepast en wordt er vermeden dat accounts van werknemers toegang hebben tot irrelevante, maar gevoelige data.

%---------- Methodologie ------------------------------------------------------
\section{Methodologie}%
\label{sec:methodologie}

In eerste instantie vindt er een literatuurstudie plaats. Hierin worden de huidige cybersecurity standaarden bekeken, voornamelijk \emph{General Data Protection Regulation (GDPR)} en \emph{Network and Information Security 2 (NIS2) Directive}. \newline

Daarnaast worden de veelvoorkomende bedreigingen voor CRM systemen onderzocht en eventuele oplossingen en best practices. Hierbij wordt ook de werking van Data Loss Prevention systemen bestudeerd.

Vervolgens helpt een requirements-analyse te bepalen aan welke functies het CRM software pakket moet voldoen. Voornamelijk houdt dit de huidige staat van de Sharepoint omgeving binnen Excentis in. Aangezien CRM ook geïmplementeerd zit in brede centrale softwarepakketten, worden deze pakketten extra onder de loep genomen om te vergelijken met de kleinere specifieke pakketten. Een voorbeeld hiervan is Odoo, die meer aanbied, terwijl Hubspot specifiek gericht is op CRM.

Aan de hand van de literatuurstudie en de requirements-analyse worden de mogelijke CRM en DLP oplossingen bekeken. \newline

De mogelijkheden worden grondig onderzocht aan de hand van officiële documentatie en mogelijke configuratie. Om incidenten te detecteren, en hierop te reageren, wordt ook onderzocht of het systeem kan integreren met JIRA ITSM, Opsgenie en MSSentinel. Wat betreft DLP oplossingen wordt er in eerste instantie gekeken of het CRM systeem zelf een DLP oplossing biedt. Mocht dit niet het geval zijn, wordt er gekeken om DLP van Microsoft Defender of Google Cloud te implementeren.

Eens een beperkt aantal tools geselecteerd zijn, wordt er een proof-of-concept van de tools opgesteld. Dit zal de samenwerking van CRM en DLP demonstreren, alsook een implementatie die voldoet aan de huidige cybersecurity standaarden. \newline

Het maken van de proof-of-concepts voor de verschillende tools duurt het langst. Dit komt aangezien de tools uitgebreid onderzocht en getest worden in verband met het implementeren van mogelijke cybersecurity maatregelen.


%---------- Verwachte resultaten ----------------------------------------------
\section{Verwacht resultaat, conclusie}%
\label{sec:verwachte_resultaten}

De meeste Customer Relationship Management tools worden verwacht een basis implementatie te hebben van cybersecurity maatregelen, zoals access control en multifactor authentication. Daarom voldoen CRM systemen vaak aan een basis overeenstemming met standaarden, waaronder GDPR en NIS2. Toch wordt verwacht een verschil op te merken tussen bepaalde tools in het niveau van overeenstemming. Waar de ene tool een basis implementatie heeft, neemt de andere geavanceerde maatregelen.

Wat betreft Data Loss Prevention, is de verwachting dat CRM systemen zelf niets bieden, maar wel samenwerken met externe tools zoals DLP van Microsoft Defender of Google Cloud. Dit komt omdat de functies van DLP een bredere toepassing hebben binnen een bedrijf dan alleen samen te werken met CRM.

Verdere maatregelen in verband met detectie, reactie en herstel zijn niet vaak te vinden binnen CRM tools zelf. Hiervoor zullen dus externe monitoring en back-up tools moeten worden gebruikt. CRM hebben wel soms zelf logging om activiteiten binnen de CRM vast te leggen.




\printbibliography[heading=bibintoc]

\end{document}